% data mahasiswa
\titleind{PERSAMAAN PROCA DAN INVARIANSI TERANYA VERSI KUATERNIONIK}
\fullname{Nur Farida Amalia} 
\idnum{123456789} 
\jurusan{Fisika}
\prodi{Pendidikan Fisika}

% data khusus tampil di cover
\gelarsarjana{Fisika}
\tahunterbitskripsi{2021} 

% data pembimbing skripsi
\pembimbingpertama{Prof. Dr. Niels Bohr} 
\nippembimbingpertama{19670320 200112 1 001}
\pembimbingkedua{Prof. Dr. Max Planck}
\nippembimbingkedua{19520101 200001 1 001}

% data penguji skripsi
\pengujipertama{Prof. Dr. Iron Man}
\nippengujipertama{19340101 200301 1 001}
\pengujikedua{Prof. Dr. Jason Bourne}
\nippengujikedua{19542010 200401 1 001}
\pengujiketiga{Prof. Albert Einstein, Ph.D.}
\nippengujiketiga{19652010 199901 1 001}
\pengujikeempat{Prof. Dr. R. P. Feynman}
\nippengujikeempat{19760101 200001 1 001}

% data nama file pustaka, abstrak dan kata pengantar
\filepustaka{pustaka.tex}
\fileabstrak{abstrak.tex}
\filekatapengantar{kata_pengantar.tex}

% data tanggal
\tanggalpernyataankeaslian{1 Januari 2021}
\tanggalpengesahan{1 Februari 2021}
\tanggalnotabimbingsatu{2 Januari 2021 }
\tanggalnotabimbingdua{3 Januari 2021}

% kata kunci untuk di halaman abstrak
\katakunci{katakunci1, katakunci2, katakunci3}
