Tapi kenyataan bagaimanapun juga harus diakui, bahwa mekanika Newton yang disokong oleh transformasi Galileo mengalami konflik dengan persamaan Maxwell. Sebagai salah satu upaya yang terbilang spektakuler untuk mendamaikan konflik antara mekanika Newton dan persamaan Maxwell dilakukan oleh Michelson dan Morley melalui serangkaian percobaan yang dilakukan dalam tahun 1887 dengan hipotesa bahwa ether itu ada. Namun Michelson-Morley pada akhirnya harus menerima kenyataan bahwa percobaannya menunjukkan bahwa ether tidak ada. Siapa sangka bahwa jalan keluar dari konflik antara mekanika Newton dan Persamaan Maxwell adalah usulan Einstein tersebut yang kini dikenal sebagai teori relativitas khusus. Teori relativitas khusus bukanlah menggantikan mekanika Newton dan transformasi Galileo melainkan mencakupi keduanya. Perhatikan di sini, apa yang dilakukan oleh Einstein adalah mengusulkan suatu postulat baru yang tentunya bertentangan dengan konsekuensi logis dari postulat yang dimiliki oleh mekanika Newton. 

Ulah Einstein dalam mengguncang dunia fisika tidak hanya sampai di situ. Pada tahun 1916, ia mengeluarkan teori relativitas umum yang bisa saya bayangkan betapa terkejoetnya para fisikawan saat itu saat mengetahuinya. Bagaimana tidak, teori relativitas umum menyatakan bahwa gravitasi bukanlah gaya, melainkan merupakan manifestasi kelengkungan ruang. Kelengkungan ruang dipengaruhi oleh massa, dan sebaliknya, pergerakan massa dipengaruhi oleh kelengkungan ruang. Mumet ga tuh! Teori relativitas umum ini mungkin bagi sebagian orang terlihat  menghabisi teori gravitasi Newton bagaikan Khabib yang menghabisi gregor. Tapi perlu diketahui, sama seperti halnya teori relativitas khusus mencakupi mekanika Newton dan transformasi Galileo, teori relativitas umum ini mencakupi teori gravitasi Newton. Jika kita hanya mencukupkan diri pada teori gravitasi Newton, mungkin kita tidak akan dapat menjelaskan fenomena pergeseran merah, dll. Bahkan mungkin para fisikawan tidak akan punya pemikiran untuk mendeteksi keberadaan gelombang gravitasi.

Proses penyusunan skripsi ini tidak lepas dari doa, bantuan,
bimbingan, motivasi dan peran dari banyak pihak. Sehingga penulis mengucapkan terimakasih kepada :
\begin{enumerate}
\item Rektor
\item Dekan
\item Ketua Prodi
\item Dosen Pembimbing
\item dll
\item Semua Pihak yang tidak dapat penulis sebutkan satu persatu yang telah memberikan kontribusi hingga selesainya skripsi ini.
\end{enumerate}
Semoga kebaikan semuanya menjadi amal ibadah yang diterima dan mendapat pahala yang berlimpah dari Allah SWT. Aamiin.

Atas segala kekurangan dan kelemahan dalam skripsi ini penulis mengharapkan saran dan kritik yang membangun.
Semoga karya tulis yang sederhana ini dapat menjadi bacaan yang bermanfaat dan dapat dikembangkan bagi peneliti-peneliti selanjutnya. 

